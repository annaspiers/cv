\begin{longtable}{@{}>{\raggedright}p{6.25in} >{\raggedleft}X@{}}

\hangindent=5ex \emph{Coding languages:} Use regularly: \texttt{R}, Use infrequently: \texttt{Python}, \texttt{MatLab}, Use rarely: \texttt{Java} & \tabularnewline

\hangindent=5ex \emph{Data manipulation and visualization in R:} \texttt{tidyverse} (\texttt{dplyr}, \texttt{ggplot2}, \texttt{tidyr}, \texttt{data.table}, \texttt{tmap}) & \tabularnewline

\hangindent=5ex \emph{GIS:} \texttt{R} (\texttt{raster}, \texttt{sf}, \texttt{lidR}), Structure from Motion photogrammetry (Agisoft Metashape), QGIS & \tabularnewline

%\hangindent=5ex \emph{Remote sensing:} Drones, multispectral sensors, FAA-licensed Remote Pilot (2017 to present) & \tabularnewline

\hangindent=5ex \emph{Inference:} Hierarchical modeling in \texttt{R} using Bayesian frameworks (\texttt{rstan}, \texttt{jagsUI}) and maximum likelihood (\texttt{lme4}), agent-based simulation modeling in Bash  & \tabularnewline

\hangindent=5ex \emph{Fieldwork:} Certified UAS pilot under University of Colorado COA, Vegetation plot establishment, Tree stem mapping using laser instruments, Tree climbing certificate from Tree Climber's International & \tabularnewline

\hangindent=5ex \emph{Version control:} git, GitHub & \tabularnewline

\hangindent=5ex \emph{Dynamic documents:} RMarkdown, \LaTeX{} & \tabularnewline

\hangindent=5ex \emph{Spoken language:} Fluent: English, Conversational: Spanish, American Sign Language \LaTeX{} & \tabularnewline

\end{longtable}
