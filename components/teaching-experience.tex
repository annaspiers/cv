\begin{longtable}{@{}>{\raggedright}p{5.25in} >{\raggedleft}X@{}}
%\emph{Lead or Co-lead Instructor} \tabularnewline

%ECL298 \texttt{R} for Data Analysis and Visualization in Science (R-DAVIS) & 2018 \tabularnewline
%\addtolength{\leftskip}{5ex}A quarter-long, 2-credit graduate course at the University of California, Davis teaching scientific computing skills (data/project management, version control, reproducible workflows using the programming language \texttt{R}) to 25+ ecologists. Adopted as part of the required curriculum for the graduate program. & \tabularnewline

%Data Carpentry Week: Introduction to \texttt{R} & 2017 \tabularnewline
%\addtolength{\leftskip}{5ex}A week-long workshop teaching scientific computing skills to 25+ learners as part of the Data Intensive Biology Summer Institute at the University of California, Davis. & \tabularnewline

%ECOL592 Introduction to \texttt{R} & 2014 \tabularnewline
%\addtolength{\leftskip}{5ex}A semester-long, 1-credit graduate course teaching data manipulation and visualization using \texttt{R} to 20+ grad students, professors, postdocs, undergraduates, and local professionals learners at Colorado State University. & \\ 
\addlinespace[1ex]

\emph{Teaching assistant}  \tabularnewline

EBIO 1240 General Biology 2, University of Colorado Boulder & 2023 \tabularnewline

EBIO 2040 Principles of Ecology, University of Colorado Boulder & 2017 - 2018, 2022 \tabularnewline

EBIO 3590 Plants and Society, University of Colorado Boulder & 2020 \tabularnewline

MATH 200 Discrete Math, Williams College & 2014 \tabularnewline

%\texttt{R} Bootcamp; University of California, Davis & 2015 \tabularnewline
%\addlinespace[1ex]

%\emph{Guest lecturer} \tabularnewline
%``Wildfire and insect outbreak effects on forest structure and composition'' CU Boulder Undergraduate Ecology. & 2020 \\ (remote) \tabularnewline
% \addlinespace[1ex]

\emph{Formal training} \tabularnewline
Inclusive Pedagogy, University of Colorado Boulder & 2018  \tabularnewline

\end{longtable}
